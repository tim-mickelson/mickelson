\documentclass[a4paper]{article}

\title{Title goes here}
\author{Tim Mickelson}
\date{05/04/2014}

\begin{document}
\maketitle

\newpage

\section{Introduction}
The purpose of this documentation is to describe the approuch of resolving the programming assigment found in the \textit{pdf} document \textit{Programmeringsuppgift} provided by Evry~\cite{pdf}.

\section{The Problem}
The programming challenge consists of reading textual documents and giving \textit{points} for documents and it's containing words. The points are then presented to the standard output as requested in the
document \textit{Programmeringsuppgift}.

\section{Rules}
The rules as interpretted by me are:
\begin{enumerate}
	\item All words shorter then 3 characters and longer then 20 will not be considered, not even when counting the ammount of words in the document.
	\item All words of the lenght 3 characters are counted but not given points.
	\item All words between the length of 4 and 10 are \textit{short words} and always given at least one point.
	\item All words between 10 and 20 letters are \textit{long words} and are encounted for only if the contain the character \textit{"-"} otherwise they are totally ignored.
	\item If the document contains more then 100 words (short and long words with \textit{"-"}) then the short words get 2 points and the long 1 point.
	\item If the document contains more then 1000 words (short and long words with \textit{"-"}) then all words with double letter get one more point.
\end{enumerate}

\section{Techincal data}
The program is compiled with JDK 1.7.0


\section{Usage}
xxx

\section{Technical Approach}

Describe classes here

\begin{thebibliography}{99}
	\bibitem{pdf}Anders Dannqvist\emph{Programmeringsuppgift} EVRY Consulting AB, 2014
\end{thebibliography}

\end{document}
